\documentclass[11pt,a4paper]{article}

%%%%%% Packages
% Handles references
\usepackage[backend=biber, style=authoryear, natbib=true]{biblatex}  
% Handle page formatting
\usepackage[margin=1in,footskip=0.25in]{geometry}
% Handle links
\usepackage[colorlinks=true, allcolors=blue]{hyperref}
% Handle lists
\usepackage{enumitem}
% Handle colors
\usepackage{xcolor}

%%%%%% Reference files
\addbibresource{references.bib}

%%%%%% Other commands
% Remove indents
\setlength{\parindent}{0pt}
% Set page spacing
\renewcommand{\baselinestretch}{1.2} 

\begin{document}

%%%%% Heading and author information

\textbf{Observations and analysis of an urban boundary layer during extreme heat episodes}
Gabriel Rios \textsuperscript{1*, 2*}, Prathap Ramamurthy \textsuperscript{1, 2}, Mark Arend \textsuperscript{2, 3}

\small{
\begin{enumerate}[leftmargin=0cm, itemsep=0mm]
	\item Department of Mechanical Engineering, CUNY City College, New York, New York
	\item NOAA Center for Earth System Sciences and Remote Sensing Technologies, New York, New York
	\item Department of Electrical Engineering, CUNY City College, New York, New York
\end{enumerate}
}

\textbf{Corresponding author}: Gabriel Rios (\href{mailto:grios001@citymail.cuny.edu}{grios001@citymail.cuny.edu})

\textbf{* Current affiliation(s)}: Department of Mechanical Engineering, CUNY City College, New York, New York; NOAA Center for Earth System Sciences and Remote Sensing Technologies, New York, New York

% Divider line
\noindent\rule{\textwidth}{1pt}

%%%%% Abstract

\section*{Abstract}

%%%%% Introduction

\section{Introduction}
Understanding the planetary boundary layer over urban areas, also called the urban boundary layer (UBL), is critical as the conditions in this layer directly affect human activity.

%%%%% Data collection and analysis

\section{Data collection and analysis}

\subsection{Observation site}
The UBL over New York City is observed and analyzed in this study.

\subsection{Observational instruments}

\subsubsection{Data availability}

\subsection{Derived quantities}

\section{Results}

%%%%% Discuss UBL turbulence
\subsection{Mean and turbulent boundary layer properties}

%%%%% Discuss the differences between normal and extreme heat days
\subsection{Normal and extreme heat boundary layer properties}

%%%%% Discuss the observed sea breeze circulation
\subsection{Effects of the sea breeze circulation}

\section{Discussion}

\section{Conclusions}

\printbibliography

\end{document}