\documentclass[11pt,a4paper]{article}

%%%%%% Packages
% Handles references
\usepackage[backend=biber, style=authoryear, natbib=true]{biblatex}  
% Handle page formatting
\usepackage[margin=1in,footskip=0.25in]{geometry}
% Handle links
\usepackage[colorlinks=true, allcolors=blue]{hyperref}
% Handle lists
\usepackage{enumitem}
% Handle colors
\usepackage{xcolor}
% Handle tables
\usepackage{tabularx, caption}

%%%%%% Reference files
\addbibresource{references.bib}

%%%%%% Other commands
% Remove indents
\setlength{\parindent}{0pt}
% Set page spacing
\renewcommand{\baselinestretch}{1.2} 

\begin{document}

%%%%% Heading and author information

\textbf{Observations and analysis of an urban boundary layer during extreme heat episodes}
Gabriel Rios \textsuperscript{1*, 2*}, Prathap Ramamurthy \textsuperscript{1, 2}, Mark Arend \textsuperscript{2, 3}

\small{
\begin{enumerate}[leftmargin=0.5cm, itemsep=0mm]
	\item Department of Mechanical Engineering, CUNY City College, New York, New York
	\item NOAA Center for Earth System Sciences and Remote Sensing Technologies, New York, New York
	\item Department of Electrical Engineering, CUNY City College, New York, New York
\end{enumerate}
}

\textbf{Corresponding author}: Gabriel Rios (\href{mailto:grios001@citymail.cuny.edu}{grios001@citymail.cuny.edu})

\textbf{* Current affiliation(s)}: Department of Mechanical Engineering, CUNY City College, New York, New York; NOAA Center for Earth System Sciences and Remote Sensing Technologies, New York, New York \\

\small{For submission to the Quarterly Journal of the Royal Meteorological Society.} \\
\small{\textit{Last updated}: \today.}

% Divider line
\noindent\rule{\textwidth}{1pt}

%%%%% Abstract

\section*{Abstract}

%%%%% Introduction

\section{Introduction}
Understanding the planetary boundary layer over urban areas, also called the urban boundary layer (UBL), is critical as the conditions in this layer directly affect human activity.

%%%%% Data collection and analysis

\section{Data collection and analysis}

\subsection{Study site}
The UBL over New York City is observed and analyzed in this study. Observational data was captured at four locations within New York City (Table \ref{tab:observation_sites}).

% Table displaying site information

\captionof{table}{Locations and details of observations sites.}
\begin{center}
	\label{tab:observation_sites}
	\begin{tabularx}{\textwidth}{l X X X X}
 		 \hline
 		  & Bronx & Manhattan & Queens & Staten Island \\
 		 \hline
 		Coordinates & 40.87248\textdegree N, -73.89352\textdegree E & 40.82044\textdegree N, -73.94836\textdegree E & 40.73433\textdegree N, -73.81585\textdegree E & 40.60401\textdegree N, -74.14850\textdegree E \\
 		Elevation (m a.g.l.) & 57.8 & 90.6 & 56.3 & 32.4 \\
	\end{tabularx}
\end{center}

\subsection{Observational instruments}
Observations of the UBL were made using a synthesis of microwave radiometers, lidars, sonic anemometers, and surface weather stations. 

Vertical profiles of temperature and vapor density were captured using microwave radiometers (Radiometrics MP-3000A). Profiles are captured at 58 height levels starting at 50 m and ending at 10 km above ground level, with vertical steps of 50 m from 50 to 500 m, 100 m from 500 m to 2 km, and 250 m steps above 2 km. 

\subsubsection{Data availability}

\subsection{Derived quantities}

\section{Results}

%%%%% Discuss UBL turbulence
\subsection{Mean and turbulent boundary layer properties}

%%%%% Discuss the differences between normal and extreme heat days
\subsection{Normal and extreme heat boundary layer properties}

%%%%% Discuss the observed sea breeze circulation
\subsection{Effects of the sea breeze circulation}

\section{Discussion}

\section{Conclusions}

\printbibliography

\end{document}